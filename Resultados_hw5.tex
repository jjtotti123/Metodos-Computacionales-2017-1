\documentclass{article}
\usepackage{verbatim}
\usepackage{graphicx}

\begin{document}
En el presente documento se quiere mostrar los resultados del analisis realizado para encontrar el tama�o de los canales ionicos para dos diferentes caso utilizando el metodo MCMC para su simulacion. Tambien, los resultados de una simulacion para la carga de un circuito RC mediante el metodo de determinacion bayesiana\\


\begin{center}
\includegraphics[scale=0.6]{graf1.png}
\end{center}
\begin{center}
\includegraphics[scale=0.6]{graf2.png}
\end{center}
Se puede observar el resultado del tama�o y ubicacion de los poros para cada caso, de tal manera que estos no toquen ninguna de las moleculas de los bordes.
\\

\begin{center}
\includegraphics[scale=0.6]{graf3.png}
\end{center}
\begin{center}
\includegraphics[scale=0.6]{graf4.png}
\end{center}
En este caso se puede observar dos histogramas que reflejan la variacion de los valores de R y C a amedida que corre la simulacion.
\\

\begin{center}
\includegraphics[scale=0.6]{graf5.png}
\end{center}
Por ultimo se observa una grafica que muetra los resultados tras el ajuste de los valores de los parametros R y C obtenidos tras la simulacion.
\\

\end{document}
