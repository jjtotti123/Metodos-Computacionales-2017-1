\documentclass{article}
\usepackage{verbatim}
\usepackage{graphicx}

\begin{document}
En el presente documento se quiere mostrar los resultados de la resolucion numerica para la ecuacion de difusion de calor en 2 dimensiones. Se plantearon dos casos de condiciones iniciales y para cada caso 3 condiciones de frontera; Abiertas,Periodicas y constantes. Entonces, se presentan 18 graficas correswpondientes a a los resultados de los casos para diferentes momentos de tiempo \\


\begin{center}
\includegraphics[scale=0.6]{t1c1a.png}
\end{center}
\begin{center}
\includegraphics[scale=0.6]{t2c1a.png}
\end{center}
\begin{center}
\includegraphics[scale=0.6]{t3c1a.png}
\end{center}
Se puede observar que para el caso 1 en condiciones abiertas la temeperatura tiende a disminuir y "aplanarse" a medida que pasa el tiempo.
\\

\begin{center}
\includegraphics[scale=0.6]{t1c1p.png}
\end{center}
\begin{center}
\includegraphics[scale=0.6]{t2c1p.png}
\end{center}
\begin{center}
\includegraphics[scale=0.6]{t3c1p.png}
\end{center}
Para el caso 1 de condiciones periodicas la tempertura se comporta de una manera mas flucutuante y en general va aumentando a medida que pasa el tiempo.
\\

\begin{center}
\includegraphics[scale=0.6]{t1c1c.png}
\end{center}
\begin{center}
\includegraphics[scale=0.6]{t2c1c.png}
\end{center}
\begin{center}
\includegraphics[scale=0.6]{t3c1c.png}
\end{center}
En el caso 1 para condiciones constante la temperatura se va "aplanando" en sentido de que tiene a aumentar pero tratando de llegar a un estado estable.
\\


\begin{center}
\includegraphics[scale=0.6]{t1c2a.png}
\end{center}
\begin{center}
\includegraphics[scale=0.6]{t2c2a.png}
\end{center}
\begin{center}
\includegraphics[scale=0.6]{t3c2a.png}
\end{center}
Se observa que en el caso 2 de condiciones abiertas la zona de la placa que tiene una temperatura mas caliente siempre se mantiene constante misntras que hacia los bordes la temperatura tiende a enfriarse.
\\

\begin{center}
\includegraphics[scale=0.6]{t1c2p.png}
\end{center}
\begin{center}
\includegraphics[scale=0.6]{t2c2p.png}
\end{center}
\begin{center}
\includegraphics[scale=0.6]{t3c2p.png}
\end{center}
En el caso 2 de condiciones periodicas la temperatura va aumentando paulatinamente mientras que las zona de la placa que tiene mayor temperatura se mantiene constante.
\\

\begin{center}
\includegraphics[scale=0.6]{t1c2c.png}
\end{center}
\begin{center}
\includegraphics[scale=0.6]{t2c2c.png}
\end{center}
\begin{center}
\includegraphics[scale=0.6]{t3c2c.png}
\end{center}
Para el caso 2 de condiciones constantes la temperatura va aumentando tratando de nivelarse a la misma temperatrura que esta mas caliente que el resto de la placa.
\\

\end{document}
